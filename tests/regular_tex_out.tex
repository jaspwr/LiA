\documentclass[a4paper, 11pt]{article}
\usepackage{amsmath}
\usepackage[utf8]{inputenc}
\usepackage{authblk}
\usepackage{tikz}
\usepackage{multicol}
\usetikzlibrary{quantikz}


\topmargin= -1.5cm
\textwidth= 18cm
\textheight= 24.5cm
\oddsidemargin= -1cm


\begin{document}    
    
    
    \subsection{Entanglement} % random comment
    Quantum entanglement is a phenomenon that occurs when two or more qubits are linked together in such a way that measuring one qubit will instantly determine the state of the other qubit.
    To put two qubits into an entangled state, we can use the following circuit:
    \begin{equation}    
        \begin{quantikz}    
            \lstick{\ket{0}} & \qw & \targ{} & \qw\\
            \lstick{\ket{0}} & \gate{H} & \ctrl{-1} & \qw\\
        \end{quantikz}
    \end{equation}The simplest way to explain this is through calculation:
    \begin{equation}    
        C H_{1} (\ket{0} \otimes \ket{0}) =
        C (
        \begin{pmatrix}    
            \frac{1}{\sqrt{2}} \\
            \frac{1}{\sqrt{2}} \\
             
        \end{pmatrix}
        \otimes
        \ket{0}
        ) = 
        \begin{pmatrix}    
            1 & 0 & 0 & 0 \\
            0 & 1 & 0 & 0 \\
            0 & 0 & 1 & 0 \\
            0 & 0 & 0 & 1 \\
             
        \end{pmatrix}
        \begin{pmatrix}    
            \frac{1}{\sqrt{2}} \\
            0 \\
            \frac{1}{\sqrt{2}} \\
            0 \\
             
        \end{pmatrix} =
        \begin{pmatrix}    
            \frac{1}{\sqrt{2}} \\
            0 \\
            0 \\
            \frac{1}{\sqrt{2}} \\
             
        \end{pmatrix}
    \end{equation}
\end{document}